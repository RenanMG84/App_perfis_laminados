\documentclass{article}%
\usepackage[T1]{fontenc}%
\usepackage[utf8]{inputenc}%
\usepackage{lmodern}%
\usepackage{textcomp}%
\usepackage{lastpage}%
\usepackage{geometry}%
\geometry{top=2cm,bottom=2cm,left=1.5cm,right=1.5cm}%
\usepackage{ragged2e}%
\usepackage{amsmath}%
%
%
%
\begin{document}%
\normalsize%
\begin{center}%
\textbf{\Large MEMORIAL DE CÁLCULO (NBR 8800/2024)}\par%
\end{center}%
Perfil: W 150 x 13,0%
\section{Cálculo de Tração}%
\label{sec:ClculodeTrao}%

%
\subsection{Cálculo da Força Resistente à Tração (Item 5.2.2)}%
\label{subsec:ClculodaForaResistenteTrao(Item5.2.2)}%
Força resistente à tração (escoamento da seção bruta):%
\[%
N_{{trd}} = \frac{{A_g \cdot f_y}}{1.10}  = \frac{{16.60 \cdot 34.50}}{1.10} = 520.64  ~  kN%
\]

%
\section{Cálculo da Força de Compressão}%
\label{sec:ClculodaForadeCompresso}%

%
\subsection{Força axial de flambagem (Item 5.3.5)}%
\label{subsec:Foraaxialdeflambagem(Item5.3.5)}%
\[%
N_{ex} = \frac{{\pi^2 \cdot E \cdot I_x}}{L_x^2} = \frac{{\pi^2 \cdot 20000 \cdot 635}}{300.00^2} = 1392.71    kN 
%
\]%
\[%
N_{ey} = \frac{{\pi^2 \cdot E \cdot I_y}}{L_y^2} = \frac{{\pi^2 \cdot 20000 \cdot 82}}{300.00^2} = 179.85    kN 
%
\]%
\[%
N_{ez} = \frac{1}{r_0^2} \cdot \left(\frac{\pi^2 \cdot E \cdot C_w}{L_z^2} + G \cdot I_t \right)  = \frac{1}{6.57^2} \cdot \left(\frac{\pi^2 \cdot 20000 \cdot 4181}{300.00^2} + 7700 \cdot 1.72 \right) = 519.79    kN 
%
\]

%
Força normal de flambagem elástica (Ne): 179.85 kN%
\subsection{Índice de esbeltez reduzido (Item 5.3.3.2)}%
\label{subsec:ndicedeesbeltezreduzido(Item5.3.3.2)}%
\[%
\lambda_0 = \sqrt{\frac{a_g \cdot f_y}{N_e}} = \sqrt{\frac{16.60 \cdot 34.50}{179.85}} = 1.78 
%
\]

%
\subsection{Fator de redução (Item 5.3.3)}%
\label{subsec:Fatordereduo(Item5.3.3)}%
\[%
\chi = \frac{0.877}{\lambda_0^2}  = \frac{0.877}{1.78^2} = 0.28 
%
\]

%
\subsection{Largura efetiva dos elementos (Item 5.3.4.2)}%
\label{subsec:Larguraefetivadoselementos(Item5.3.4.2)}%
Esbeltez da alma:%
\[%
\frac{b}{t} = \frac{11.80}{0.43} = 27.44 
%
\]%
Esbeltez limite da alma:%
\[%
\frac{1.49 \cdot \sqrt{\frac{E}{f_y}}}{\sqrt{\chi}}  = \frac{1.49 \cdot \sqrt{\frac{20000}{34.50}}}{\sqrt{0.28}}  = 68.36 
%
\]%
Largura efetiva da alma:%
\[%
 b_{ef} = 11.80 \ cm%
\]%
Esbeltez da mesa:%
\[%
\frac{b}{t} = \frac{5.00}{0.49} = 10.20 
%
\]%
Esbeltez limite da mesa:%
\[%
\frac{0.56 \cdot \sqrt{\frac{E}{f_y}}}{\sqrt{\chi}}  = \frac{0.56 \cdot \sqrt{\frac{20000}{34.50}}}{\sqrt{0.28}}  = 25.69 
%
\]%
Largura efetiva da mesa:%
\[%
 b_{ef} = 5.00 \ cm%
\]%
Área efetiva:%
\[%
 A_{ef} = 16.60 \ {cm}^2%
\]%
Força axial resistente de cálculo:%
\[%
 N_{c,rd} = \frac{\chi \cdot A_{ef} \cdot f_y}{1.10}  = \frac{0.28 \cdot 16.60 \cdot 34.50}{1.10}  = 143.386   kN 
%
\]

%
\section{Cálculo da Força Cortante}%
\label{sec:ClculodaForaCortante}%

%
\subsection{Cálculo da Força Resistente a Cortante em X (Item 5.4.3.1)}%
\label{subsec:ClculodaForaResistenteaCortanteemX(Item5.4.3.1)}%
Área efetiva de cisalhamento:%
\[%
A_w = d' \cdot t_w = (11.80 \cdot 0.43) = 5.07    {cm}^2%
\]%
Força cortante de plastificação:%
\[%
V_{pl} = 0.6 \cdot A_w \cdot f_y = 0.6 \cdot 5.07 \cdot 34.50 = 105.03    kN%
\]%
Esbeltez do perfil:%
\[%
\lambda = \frac{d'}{t_w} = \frac{11.80}{0.43} = 27.44%
\]%
Lambda P:%
\[%
\lambda_p = 1.10 \cdot \sqrt{\frac{{k_v \cdot E}}{{f_y}}} = 1.10 \cdot \sqrt{\frac{{5 \cdot 20000}}{{34.50}}} = 59.22 %
\]%
Lambda R:%
\[%
\lambda_r = 1.37 \cdot \sqrt{\frac{{k_v \cdot E}}{{f_y}}} = 1.37 \cdot \sqrt{\frac{{5 \cdot 20000}}{{34.50}}} = 73.76 %
\]%
Força cortante resistente:%
\[%
V_{rd} = \frac{V_{pl}}{1.10}  = \frac{105.03}{1.10} = 95.48   ~   kN%
\]

%
\subsection{Cálculo da Força Resistente a Cortante em Y (Item 5.4.3.5)}%
\label{subsec:ClculodaForaResistenteaCortanteemY(Item5.4.3.5)}%
Área efetiva de cisalhamento:%
\[%
A_w = 2 \cdot b_f \cdot t_f = 2 \cdot 10.00 \cdot 0.49  = 9.80    {cm}^2%
\]%
Força cortante de plastificação:%
\[%
V_{pl} = 0.6 \cdot A_w \cdot f_y = 0.6 \cdot 9.80 \cdot 34.50 = 202.86    kN%
\]%
Valor de 'h':%
\[%
h = \frac{b_f}{2} = \frac{10.00}{2} = 5.00 \ cm%
\]%
Esbeltez do perfil:%
\[%
\lambda = \frac{h}{t_w} = \frac{5.00}{0.43} = 11.63%
\]%
Lambda P:%
\[%
\lambda_p = 1.10 \cdot \sqrt{\frac{{k_v \cdot E}}{{f_y}}} = 1.10 \cdot \sqrt{\frac{{1.2 \cdot 20000}}{{34.50}}} = 29.01%
\]%
Lambda R:%
\[%
\lambda_r = 1.37 \cdot \sqrt{\frac{{k_v \cdot E}}{{f_y}}} = 1.37 \cdot \sqrt{\frac{{1.2 \cdot 20000}}{{34.50}}} = 36.13 %
\]%
Força cortante resistente:%
\[%
V_{rd} = \frac{V_{pl}}{1.10}  = \frac{202.86}{1.10} = 184.42   ~   kN%
\]

%
\section{Cálculo do Momento Fletor}%
\label{sec:ClculodoMomentoFletor}%

%
\subsection{Cálculo do momento fletor resistente em X (Item D.2.1)}%
\label{subsec:ClculodomomentofletorresistenteemX(ItemD.2.1)}%
Momento fletor de plastificação: %
\[%
M_{pl} = z_x \cdot f_y = 96.40 \cdot 34.50 = 3325.80 \  \text{kN} \cdot \text{cm}%
\]%
Flambagem local da alma {-} FLA:%
\[%
\lambda = \frac{d'}{t_w} = \frac{11.80}{0.43}  = 27.44 %
\]%
Lambda P:%
\[%
\lambda_p = 3.76 \cdot \sqrt{\frac{E}{f_y}} = 3.76 \cdot \sqrt{\frac{20000}{34.50}} = 90.53%
\]%
Lambda R:%
\[%
\lambda_r = 5.70 \cdot \sqrt{\frac{E}{f_y}} = 5.70 \cdot \sqrt{\frac{20000}{34.50}} = 137.24%
\]%
Momento fletor resistente:%
\[%
M_{rd} = \frac{M_{pl}}{1.10} = \frac{3325.80}{1.10} = 3023.45 \  \text{kN} \cdot \text{cm}%
\]%
Flambagem local da mesa {-} FLM:%
\[%
\lambda = \frac{b_f}{2 \cdot t_f} = \frac{10.00}{0.49}  = 10.20 %
\]%
Lambda P:%
\[%
\lambda_p = 0.38 \cdot \sqrt{\frac{E}{f_y}} = 0.38 \cdot \sqrt{\frac{20000}{34.50}} = 9.15%
\]%
Lambda R:%
\[%
\lambda_r = 0.83 \cdot \sqrt{\frac{E}{0.7 \cdot f_y}} = 5.70 \cdot \sqrt{\frac{20000}{0.7 \cdot 34.50}} = 23.89%
\]%
Momento fletor resistente:%
\[%
M_{rd} = \frac{1}{1.10} \cdot \left( M_{pl} - \left( M_{pl} - M_r \right) \cdot \frac{\lambda - \lambda_p}{\lambda_r - \lambda_p} \right)  = \frac{1}{1.10} \cdot \left( 3325.80 - \left( 3325.80 - 2072.07 \right) \cdot \frac{10.20 - 9.15}{23.89 - 9.15} \right)  = 2941.88 \  \text{kN} \cdot \text{cm} %
\]%
Flambagem lateral com torção {-} FLT:%
\[%
\lambda = \frac{L_b}{r_y} = \frac{300.00}{2.22} = 135.14 %
\]%
Lambda P:%
\[%
\lambda_p = 1.76 \cdot \sqrt{\frac{E}{f_y}} = 1.76 \cdot \sqrt{\frac{20000}{34.50}} = 42.38%
\]%
Lambda R:%
\[%
\beta_1 = \frac{ \left( f_y - \sigma_r \right) \cdot w_x}{E \cdot i_t} = \frac{ \left( 34.50 - 0.3 \cdot 34.50 \right) \cdot 85.80}{20000.00 \cdot 1.72} = 0.06 %
\]%
\[%
\lambda_r = \frac{1.38 \cdot C_b \sqrt{I_y \cdot I_t}}{r_y \cdot I_t \cdot \beta_1} \cdot \sqrt{1 + \sqrt{1 + \frac{27 \cdot C_w \cdot \beta_1^2}{C_b^2 \cdot I_y}}} = \frac{1.38 \cdot 1.00 \sqrt{82.00 \cdot 1.72}}{2.22 \cdot 1.72 \cdot 0.06} \cdot \sqrt{1 + \sqrt{1 + \frac{27 \cdot 4181 \cdot 0.06^2}{1.00^2 \cdot 82.00}}} = 132.32 %
\]%
Momento fletor resistente:%
\[%
M_{cr} = \frac{C_b \cdot \pi^2 \cdot E \cdot I_y}{L_b^2} \cdot \sqrt{ \frac{C_w}{I_y} \cdot \left( 1 + 0.0039 \cdot \frac {I_t \cdot L_b^2}{C_w}\right)} = \frac{1.00 \cdot \pi^2 \cdot 20000.00 \cdot 82.00}{300.00^2} \cdot \sqrt{ \frac{4181}{82.00} \cdot \left( 1 + 0.0039 \cdot \frac {1.72 \cdot 300.00^2}{4181}\right)}  = 1248.91 \  \text{kN} \cdot \text{cm} %
\]

%
\subsection{Cálculo do momento fletor resistente em Y (Item D.2.1)}%
\label{subsec:ClculodomomentofletorresistenteemY(ItemD.2.1)}%
Flambagem local da alma {-} FLA:%
\[%
\lambda = \frac{d'}{t_w} = \frac{11.80}{0.43}  = 27.44 %
\]%
Lambda P:%
\[%
\lambda_p = 1.12 \cdot \sqrt{\frac{E}{f_y}} = 1.12 \cdot \sqrt{\frac{20000}{34.50}} = 26.97%
\]%
Lambda R:%
\[%
\lambda_r = 1.40 \cdot \sqrt{\frac{E}{f_y}} = 1.40 \cdot \sqrt{\frac{20000}{34.50}} = 33.71%
\]%
Momento fletor resistente:%
\[%
M_{rd} = \frac{1}{1.10} \cdot \left( M_{pl} - \left( M_{pl} - M_r \right) \cdot \frac{\lambda - \lambda_p}{\lambda_r - \lambda_p} \right)  = \frac{1}{1.10} \cdot \left( 879.75 - \left( 879.75 - 565.80 \right) \cdot \frac{27.44 - 26.97}{33.71 - 26.97} \right)  = 779.65 \  \text{kN} \cdot \text{cm} %
\]%
Flambagem local da mesa {-} FLM:%
\[%
\lambda = \frac{b_f}{2 \cdot t_f} = \frac{10.00}{0.49}  = 10.20 %
\]%
Lambda P:%
\[%
\lambda_p = 0.38 \cdot \sqrt{\frac{E}{f_y}} = 0.38 \cdot \sqrt{\frac{20000}{34.50}} = 9.15%
\]%
Lambda R:%
\[%
\lambda_r = 0.83 \cdot \sqrt{\frac{E}{0.7 \cdot f_y}} = 0.83 \cdot \sqrt{\frac{20000}{0.7 \cdot 34.50}} = 23.89%
\]%
Momento fletor resistente:%
\[%
M_{rd} = \frac{1}{1.10} \cdot \left( M_{pl} - \left( M_{pl} - M_r \right) \cdot \frac{\lambda - \lambda_p}{\lambda_r - \lambda_p} \right)  = \frac{1}{1.10} \cdot \left( 879.75 - \left( 879.75 - 396.06 \right) \cdot \frac{10.20 - 9.15}{23.89 - 9.15} \right)  = 768.30 \  \text{kN} \cdot \text{cm} %
\]

%
\end{document}